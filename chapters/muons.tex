\chapter{Muons in ATLAS}
  
\textit{I really need to find a quote for this chapter}
\vspace{5mm}
\begin{flushright}
--- Miha Zgubi\v{c}
\end{flushright}

\thispagestyle{empty}
\newpage
The ATLAS detector, described in the previous chapter, is a monument
to the engineers and physicists who designed and built it. However,
in order to turn its raw output of nearly one hundred million readout
channels per event to physics knowledge, it requires an intermediate step:
the reconstruction of physics objects.

Muons, the central physics object in this thesis, are reconstructed
by combining the information from the tracker and the muon spectrometer.
In order to be able to do precision studies the Monte Carlo (MC)
simulation of the detector response needs to be calibrated to match
the response in the real detector. In particular, the transverse momentum of
muons needs to be calibrated to the resolution obtained in the real
detector. Similarly, the reconstruction, track-to-vertex-association (TTVA),
and isolation efficiencies need to be calibrated to match those in 
recorded data. These calibrations are obtained through measurements
and carry associated uncertainties.

This chapter describes the first novel contributions from the author.
The first is an improved method of background subtraction in the isolation
efficiency measurements, which significantly reduces the systematic
uncertainty for muons with $\pt < 15$ \GeV. The second is an attempt
to improve muon momentum resolution.

\section{Reconstruction}

Track reconstruction in ID. vertex finding.

Muon reconstruction algorithms, muon types.

Efficiencies, scale factors, tag and probe method.
\cite{Aad:2016jkr}

\section{Momentum calibration}

Things that affect the resolution.

\section{Isolation}

\section{Isolation background subtraction}

\section{VADER4mu}

\section{Other objects}
