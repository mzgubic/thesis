\chapter{Muons in ATLAS}
  
\textit{I really need to find a quote for this chapter}
\vspace{5mm}
\begin{flushright}
--- Miha Zgubi\v{c}
\end{flushright}

\thispagestyle{empty}
\newpage
The ATLAS detector, described in the previous chapter, is a monument
to the engineers and physicists who designed and built it. However,
in order to turn its raw output of nearly one hundred million readout
channels per event to physics knowledge, it requires an intermediate step:
the reconstruction of physics objects.

Muons, the central physics object in this thesis, are reconstructed
by combining the information from the tracker and the muon spectrometer.
In order to be able to do precision studies the Monte Carlo (MC)
simulation of the detector response needs to be calibrated to match
the response in the real detector. In particular, the transverse momentum of
muons needs to be calibrated to the resolution obtained in the real
detector. Similarly, the reconstruction, track-to-vertex-association (TTVA),
and isolation efficiencies need to be calibrated to match those in 
recorded data. These calibrations are obtained through measurements
and carry associated uncertainties.

This chapter describes the first novel contributions from the author.
The first is an improved method of background subtraction in the isolation
efficiency measurements, which significantly reduces the systematic
uncertainty for muons with $\pt < 15$ \GeV. The second is an attempt
to improve muon momentum resolution.

\section{Reconstruction}

Tracks in the inner detector are reconstructed indiscriminately for
all charged particles. First, raw data from the detectors is converted
into space-points which form the basis of tracking. The main track finding
algorithm proceeds inside-out by finding the seeds in the silicon layers.
The seeds are then combined into roads and the extension to the TRT layer
is probed to add hits in the outermost layer. The final collection of hits
is fit to obtain the track parameters \cite{ATLAS-CONF-2010-072, Cornelissen:1020106}.
Tracks with at least $\pt > 400$ \MeV~and some other requirements, found
in Ref. \cite{ATL-PHYS-PUB-2015-026} are then used to find the primary
vertices. The reconstruction of primary vertices proceeds by first finding
the vertices and then fitting them \cite{Aaboud:2016rmg}. Finally, tracks
are associated with vertices which allows the computation of $d_0$, the
transverse impact parameter, which quantifies the shortest distance
between the track and the beam-line, and $z_0$, the longitudinal impact
parameter, which quantifies the difference between the $z$ coordinate
of the associated primary vertex, and the $z$ coordinate of the point
where $d_0$ is computed.


Muon reco in MS, combination.

Muon reconstruction algorithms, muon types.

\cite{ILLINGWORTH198887} % hough transform

Efficiencies, scale factors, tag and probe method.
\cite{Aad:2016jkr} % muons

\section{Momentum calibration}

Things that affect the resolution.

\section{Isolation}

\section{Isolation background subtraction}

\section{VADER4mu}

\section{Other objects}
