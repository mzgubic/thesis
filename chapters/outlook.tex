\chapter{Outlook}

\textit{"Prediction is very difficult, especially about the future."}

\vspace{5mm}
\begin{flushright}
--- Niels Bohr
\end{flushright}

\thispagestyle{empty}

\newpage

Sensitivity projections are an essential piece of information for
making informed choices about potential future projects the community
should undertake.
It is therefore important to assess what can be achieved with the LHC
machine, and in particular what the physics reach of the full dataset
will be. The dataset is expected to be recorded at 14 \TeV~and to
correspond to 3000 $\ifb$ of integrated luminosity, and is known
as the High-Luminosity LHC programme (HL-LHC).

\section{Sensitivity extrapolation}

As a part of the community-wide projections project \cite{ATL-PHYS-PUB-2018-054, Cepeda:2019klc}
an extrapolation of results obtained from the then state-of-the-art
analysis, described in Ref. \cite{ATLAS-CONF-2018-026}, was made.
While the general approach of the analysis used in the extrapolation
is not too different from the one described in this thesis, a number
of important improvements have been made, most notably in categorisation
and background modelling, resulting in approximately 25\% improvement
in sensitivity.

Apart from the increase in integrated luminosity from 80 to 3000 $\ifb$
increases in production cross-sections from 13 to 14 \TeV~were also
taken into account. The signal yields were scaled according to the
values in Ref. \cite{deFlorian:2016spz}, separately for the $\ggf$
and VBF production modes. The background yields were scaled according
to the parton luminosity ratio as described in Ref. \cite{Heinemeyer:2013tqa},
taking into account that the backgrounds are produced predominantly via
the interaction of quarks.

The performance of the reconstruction of physics objects is assumed
to remain unchanged to simplify the extrapolation. While the higher
\pileup~is likely to degrade performance this is expected to be
roughly balanced by the improved performance of the ATLAS detector.
Muon resolution is an exception to this rule because of the
expected improvements in performance of the ATLAS Inner Tracker (ITk)
upgrade. To model this improvement the signal widths were reduced 
by 15--30\%, depending on the analysis category.




mention:
- object performance: improvements cancel out higher lumi
- systematic uncertainties two scenarios: S1 (same) and S2 (improved)
    - assume zero spurious signal uncertainty
    - halve theory uncertainties
    - luminosity uncertainty at 1\%
- ITk improvements expected from Ref. [], improve resolution by 15-30\%


