\chapter{Outlook}

\textit{"Prediction is very difficult, especially about the future."}

\vspace{5mm}
\begin{flushright}
--- Niels Bohr
\end{flushright}

\thispagestyle{empty}

\newpage

Sensitivity projections are an essential piece of information for
making informed choices about potential future projects the community
should undertake.
It is therefore important to assess what can be achieved with the LHC
machine, and in particular what the physics reach of the full dataset
will be. The dataset is expected to be recorded at 14 \TeV~and to
correspond to 3000 $\ifb$ of integrated luminosity, and is known
as the High-Luminosity LHC programme (HL-LHC).

As a part of the community-wide projections project \cite{ATL-PHYS-PUB-2018-054, Cepeda:2019klc}
an extrapolation of results described in Ref. \cite{ATLAS-CONF-2018-026}
was made. At the time, this was the state-of-the-art analysis.
While the general approach of the analysis used in the extrapolation
is not too different from the one described in this thesis, a number
of important improvements have been developed, most notably in categorisation
and background modelling, resulting in approximately 25\% improvement
in sensitivity of the analysis.

Apart from the increase in integrated luminosity from 80 to 3000 $\ifb$
increases in production cross-sections from 13 to 14 \TeV~are also
taken into account. The signal yields are scaled according to the
values in Ref. \cite{deFlorian:2016spz}, separately for the $\ggf$
and VBF production modes. The background yields are scaled according
to the parton luminosity ratio as described in Ref. \cite{Heinemeyer:2013tqa},
taking into account that the backgrounds are produced predominantly via
the interaction of quarks.

The performance of the reconstruction of physics objects is assumed
to remain unchanged to simplify the extrapolation. While the higher
\pileup~is likely to degrade performance this is expected to be
roughly balanced by the improved performance of the ATLAS detector.
Muon resolution is an exception to this rule because of the
expected improvements in performance of the ATLAS Inner Tracker (ITk)
upgrade. To model this improvement the signal widths are reduced 
by 30\% in the VBF and Central analysis categories and by 15\% in
the Forward categories to emulate the expected improvements
\cite{Collaboration:2285585}.

Two scenarios are considered regarding the treatment of systematic
uncertainties. Scenario 1 (S1) assumes the relative systematic
uncertainties remain as they are today, while Scenario 2 (S2)
halves most of the theoretical uncertainties on the signal normalisation
are halved, with the exception of the PDF uncertainties. The
spurious signal uncertainty is assumed to be negligible in both S1
and S2, while the luminosity uncertainty is set at 1\%.

Table \ref{tab:out:res} shows the comparison of expected uncertainty
on the signal strength for the Run 2 analysis with 79.8 $\ifb$
and the S1 and S2 scenarios at the HL-LHC. It can be seen that 
even with the full HL-LHC dataset the analysis will be limited
by the statistical uncertainty on data. On top of these results,
an additional 25\% improvement in sensitivity can be expected
from the improved background modelling and categorisation
presented in this thesis.

\begin{table}[htb]
  \renewcommand{\arraystretch}{1.3}
  \centering
  \caption{
    The expected uncertainties on the signal strength measurement with
    the Run 2, 79.8 $\ifb$ dataset and the S1 and S2 systematic
    uncertainty scenarios for the HL-LHC dataset. Columns show
    the total uncertainty, statistical uncertainty, experimental
    systematic uncertainty, and the uncertainty on the signal
    normalisation.
    Reproduced from Ref. \cite{ATL-PHYS-PUB-2018-054}.}
  \label{tab:out:res}
  \begin{tabular}{ccccc}
    \toprule
    \midrule
    Scenario & $\Delta_\text{tot}/\sigma_\text{SM}$ 
             & $\Delta_\text{stat}/\sigma_\text{SM}$ 
             & $\Delta_\text{exp}/\sigma_\text{SM}$ 
             & $\Delta_{\mu_\text{sig}}/\sigma_\text{SM}$ \\
    \midrule
    Run 2, 79.8 $\ifb$ & $^{+1.04}_{-1.06}$ & $^{+0.99}_{-1.03}$ & $^{+0.03}_{-0.03}$ & $^{+0.32}_{-0.27}$ \\
    HL-LHC S1          & $^{+0.15}_{-0.14}$ & $^{+0.12}_{-0.13}$ & $^{+0.03}_{-0.03}$ & $^{+0.08}_{-0.05}$ \\
    HL-LHC S2          & $^{+0.13}_{-0.13}$ & $^{+0.12}_{-0.13}$ & $^{+0.02}_{-0.02}$ & $^{+0.05}_{-0.04}$ \\
    \midrule
    \bottomrule
  \end{tabular}
\end{table}

This is in good agreement with a previous study of $\hmumu$ prospects
\cite{ATL-PHYS-PUB-2018-006} based on generator-level MC simulation.
The study assumed a dataset equivalent to 3000 $\ifb$ collected at 14
\TeV, the average number of interactions per bunch crossing
$\langle \mu \rangle = 200$, and the most up-to-date detector performance.
The projected accuracy of the signal strength measurement was found to
be 13\%, while the discovery significance was found to be larger than
$9\sigma$.

Additionally, it is worth presenting a simple extrapolation of the results
presented in this thesis to the near future. The reason is that the
analysis is approaching the significance threshold considered as 
evidence (3 $\sigma$), and that a considerable improvement of $\sim 25\%$
has been made over the iteration of the analysis on which the more detailed
study was based. Assuming that the discovery significance as well as the
uncertainty on the signal strength scale with luminosity as
$1/\sqrt{\mathcal{L}}$, which is reasonable for the near future,
the 3 $\sigma$ benchmark will be reached with 560 $\ifb$, and the 5 $\sigma$
benchmark with 1550 $\ifb$, while the 0.5 uncertainty on the signal strength
will be reached with 275 $\ifb$, and 0.3 with 760 $\ifb$.

At the time of writing this thesis the latest result from the CMS experiment
is using 35.9 $\ifb$ of data collected in 2016 at the centre-of-mass energy
of 13 \TeV, and in combination with the earlier data collected at 7 and 8 \TeV~results
in the observed (expected) significance of 0.9 (1.0) $\sigma$.
In comparison to ATLAS, the CMS experiment benefits from a larger
magnetic field, translating directly to better muon resolution.
As a result, its performace is expected to be superior to that of ATLAS
given the equivalent dataset, meaning that CMS is therefore expected to play an
important role in the future studies of the Higgs boson decay to a pair
of muons.






