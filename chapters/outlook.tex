\chapter{Outlook}

\textit{"Prediction is very difficult, especially about the future."}

\vspace{5mm}
\begin{flushright}
--- Niels Bohr
\end{flushright}

\thispagestyle{empty}

\newpage

Sensitivity projections are an essential piece of information for
making informed choices about potential future projects the community
should undertake.
It is therefore important to assess what can be achieved with the LHC
machine, and in particular what the physics reach of the full dataset
will be. The dataset is expected to be recorded at 14 \TeV~and to
correspond to 3000 $\ifb$ of integrated luminosity, and is known
as the High-Luminosity LHC programme (HL-LHC).

\section{Sensitivity extrapolation}

As a part of the community-wide projections project \cite{ATL-PHYS-PUB-2018-054, Cepeda:2019klc}
an extrapolation of results obtained from the then state-of-the-art
analysis, described in Ref. \cite{ATLAS-CONF-2018-026}, was made.
While the general approach of the analysis used in the extrapolation
is not too different from the one described in this thesis, a number
of important improvements have been made, most notably in categorisation
and background modelling, resulting in approximately 25\% improvement
in sensitivity.





mention:
- change in xsec between 13 and 14 TeV (cite signal and background calcualtions separately)
- object performance: improvements cancel out higher lumi
- systematic uncertainties two scenarios: S1 (same) and S2 (improved)
    - assume zero spurious signal uncertainty
    - halve theory uncertainties
    - luminosity uncertainty at 1\%
- ITk improvements expected from Ref. [], improve resolution by 15-30\%


