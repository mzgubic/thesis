\chapter{Outlook}

\textit{"Prediction is very difficult, especially about the future."}

\vspace{5mm}
\begin{flushright}
--- Niels Bohr
\end{flushright}

\thispagestyle{empty}

\newpage

Sensitivity projections are an essential piece of information for
making informed choices about potential future projects the community
should undertake.
It is therefore important to assess what can be achieved with the LHC
machine, and in particular what the physics reach of the full dataset
will be. The dataset is expected to be recorded at 14 \TeV~and to
correspond to 3000 $\ifb$ of integrated luminosity, and is known
as the High-Luminosity LHC programme (HL-LHC).

\section{Sensitivity extrapolation}

As a part of the community-wide projections project \cite{ATL-PHYS-PUB-2018-054, Cepeda:2019klc}
an extrapolation of results described in Ref. \cite{ATLAS-CONF-2018-026}
was made. At the time, this was the state-of-the-art analysis.
While the general approach of the analysis used in the extrapolation
is not too different from the one described in this thesis, a number
of important improvements have been developed, most notably in categorisation
and background modelling, resulting in approximately 25\% improvement
in sensitivity of the analysis.

Apart from the increase in integrated luminosity from 80 to 3000 $\ifb$
increases in production cross-sections from 13 to 14 \TeV~are also
taken into account. The signal yields are scaled according to the
values in Ref. \cite{deFlorian:2016spz}, separately for the $\ggf$
and VBF production modes. The background yields are scaled according
to the parton luminosity ratio as described in Ref. \cite{Heinemeyer:2013tqa},
taking into account that the backgrounds are produced predominantly via
the interaction of quarks.

The performance of the reconstruction of physics objects is assumed
to remain unchanged to simplify the extrapolation. While the higher
\pileup~is likely to degrade performance this is expected to be
roughly balanced by the improved performance of the ATLAS detector.
Muon resolution is an exception to this rule because of the
expected improvements in performance of the ATLAS Inner Tracker (ITk)
upgrade. To model this improvement the signal widths are reduced 
by 15--30\% \cite{Collaboration:2285585}, depending on the analysis
category.

Two scenarios are considered regarding the treatment of systematic
uncertainties. Scenario 1 (S1) assumes the relative systematic
uncertainties remain as they are today, while Scenario 2 (S2)
halves most of the theory uncertainties on the signal normalisation
are halved, with the exception of the PDF uncertainties. The
spurious signal uncertainty is assumed to be negligible in both S1
and S2, while the luminosity uncertainty is set at 1\%.

Table \ref{tab:out:res} shows the comparison of expected uncertainty
on the signal strength for the Run 2 analysis with 79.8 $\ifb$
and the S1 and S2 scenarios at the HL-LHC.





