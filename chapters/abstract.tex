%\chapter*{Abstract}

\begin{abstract}
The Standard Model predicts the Higgs boson to interact
with all massive particles. At present, experimental
evidence exists for the interactions with bosons and 
third generation fermions. This thesis focuses on the
search for the decay of the Higgs boson to muon pairs,
the most sensitive way of probing the interactions with
second generation fermions. The search uses proton-proton
collision data collected at the centre-of-mass energy
$\sqrts = 13$ \TeV~collected by the ATLAS experiment
between 2015 and 2018 and corresponds to an integrated
luminosity of 140 $\ifb$. The analysis selects events
with oppositely charged muon pairs and splits them in
categories using a gradient boosting classifier to maximise
the sensitivity. A simultaneous maximum likelihood fit
is performed in all analysis categories to extract the
signal strength and evaluate the compatibility of the
data with the background-only hypothesis. The signal strength
is measured to be $0.5 \pm 0.7$, where the uncertainty is
dominated by the statistical uncertainty on the data.
The observed (expected) significance is found to be
0.8 (1.5) $\sigma$. The sensitivity projection for the 3000
$\ifb$ dataset collected at 14 \TeV~estimates that
the signal strength will be measured to 13\% 
and still limited by the statistical uncertainty on the
data.
\end{abstract}


%\thispagestyle{empty}
