\chapter*{Preface}

Modern particle physics relies on the collaboration of large
numbers of scientists to build and operate the accelerators and
detectors, analyse the data, and make sense of it in the
context of a theory of fundamental interactions. Chapters 1
and 2 review the theoretical backdrop and the experimental
setup, both a result of such collaboration, and both necessary
prerequisites for the entirety of the original work presented
in this thesis. My original contributions and academic
activities during the DPhil are presented in the following
text, some of which is described in more detail in Chapters 3
4, and 5 of the thesis.

\section*{Muon performance}

Chapter 3 first describes how muons are reconstructed,
identified, and calibrated in ATLAS, which was done by
collaborators. My constributions include the development
of the new background subtraction method and the work on
improving muon resolution.

The broad idea for the novel background subtraction
based on template fit originated from a collaborator and 
was designated as my qualification task for ATLAS authorship.
However, the following contributions were my own:
\begin{itemize}
\item Implementation and developement of the template fit background
subtraction method, and its validation by producing a toy
dataset with known efficiencies from MC simulated events.
\item Implementation of the computation of isolation scale factors in
the common framework of Muon Combined Performance (MCP) group.
\item The performance plots of the improved background
subtraction method on 2016 data \cite{Zgubic:2293041}.
\item The isolation part of early muon performance
studies for the 2017 dataset \cite{Bellomo:2282672}.
\item The isolation part of muon performance
studies for data collected at high number of mean interactions
per crossing \cite{Kohler:2293040}.
\item Validation of the performance of the new isolation
selections designed to be robust with respect to high
\pileup~\cite{Zgubic:2320874}.
\item Description of the background subtraction method in
the MCP paper [currently in internal review].
\end{itemize}

The work on VADER4$\mu$ was done with a collaborator who
prepared the datasets and made the figure of the resolution
in various $\hmumu$ categories. My own work included:
\begin{itemize}
\item Designing the machine learning approach using 
gradient boosting regressors and the cross-validation
pipeline.
\item Performing the hyperparameter search and feature
selection for both types of regression models, and
their validation on the dimuon datasets.
\end{itemize}

\section*{$\hmumu$ analysis}




\section*{Additional academic activities}



MLHEP 2018
Tutorial on using the method at MLHEP 2019
Keble AI symposium
ML+physics seminar series

















