%\chapter*{Preface}
\begin{originality}

Modern particle physics relies on international collaboration
of large numbers of scientists to build and operate the accelerators and
detectors, analyse the data, and make sense of it in the
context of a theory of fundamental interactions. Chapters 1
and 2 review the theoretical backdrop and the experimental
setup, both a result of such collaboration, and both necessary
prerequisites for the entirety of the original work presented
in this thesis. My original contributions and academic
activities during the DPhil are presented in the following
text, some of which is described in more detail in Chapters 3
4, and 5 of the thesis.

\subsection*{Muon performance}

Chapter 3 first describes how muons are reconstructed,
identified, and calibrated in ATLAS, which was done by
collaborators. My constributions include the development
of the new background subtraction method and the work on
improving muon resolution.

The broad idea for the novel background subtraction
based on template fit originated from a collaborator and 
was designated as my qualification task for ATLAS authorship.
However, the following contributions were my own:
\begin{itemize}
\item Implementation and developement of the template fit background
subtraction method, and its validation by producing a toy
dataset with known efficiencies from MC simulated events.
\item Implementation of the computation of isolation scale factors in
the common framework of Muon Combined Performance (MCP) group.
\item The performance plots of the improved background
subtraction method on 2016 data \cite{Zgubic:2293041}.
\item The isolation part of early muon performance
studies for the 2017 dataset \cite{Bellomo:2282672}.
\item The isolation part of muon performance
studies for data collected at high number of mean interactions
per crossing \cite{Kohler:2293040}.
\item Validation of the performance of the new isolation
selections designed to be robust with respect to high
\pileup~\cite{Zgubic:2320874}.
\item Description of the background subtraction method in
the MCP paper [in preparation].
\end{itemize}

\noindent The work on VADER4$\mu$ was done with a collaborator who
prepared the datasets and made the figure of the resolution
in various $\hmumu$ categories. My own work included:
\begin{itemize}
\item Designing the machine learning approach using 
gradient boosting regressors and the cross-validation
pipeline.
\item Performing the hyperparameter search and feature
selection for both types of regression models, and
their validation on the dimuon datasets.
\end{itemize}

\subsection*{$\hmumu$ analysis}

I was a key member of the analysis team and directly
contributed to four publications. A list of contributions
is shown below for each of them.

\noindent Conference note at ICHEP 2018 conference \cite{ATLAS-CONF-2018-026}:
\begin{itemize}
\item An extensive review of the analysis selection, including the study
of including Loose muons up to 2.7 in $|\eta|$ and their resolution,
using the sagitta bias correction to the muon momentum to reduce
the effects of misalignment in the inner tracker, removing the $\pt$
requirement on the subleading muon, the use of a looser isolation
selection, the use of forward jet vertex tagging, and the suitability
of different generators for the Drell-Yan backgrounds.
\item Initial studies arguing for improved and expanded machine
learning approach to event categorisation.
\item Implementation of the analysis selection and plotting software
pacakges, which became the analysis standard, and the implementation
of the categorisation including the machine learning approach.
\item Preparation of figures and tables for the publication and
co-authorship of the supporting documentation.
\item Implementation of a number of studies required during the
ATLAS internal review.
\item Presentation of the analysis at the ATLAS internal unblinding
approval, and the collaboration approval meetings.
\end{itemize}

\noindent Conference note at EPS 2019 conference \cite{ATLAS-CONF-2019-028}:
\begin{itemize}
\item Proposal of new techniques that were used in the
published approach, namely the use of k-fold cross-validation,
and the harmonisation of classifier outputs by transforming the
outputs to a uniform distribution.
\item Studies of input variables, boundary optimisation and the
associated bias, the study of training dataset size and the potential
of data augmentation, as well as study a study of resolution in
the analysis categories.
\item A study of a deep neural network approach trained in an
adversarial way in order to remove the dependence on the invariant
mass while using all available inputs.
\item A study of the background mass shaping introduced by the
machine learning selection.
\end{itemize}

\noindent Journal publication [in preparation].
\begin{itemize}
\item Validation of the kinematic distributions and categorisation
scheme, as well as the per-category resolution validation.
\end{itemize}

\noindent Projections for HL-LHC \cite{ATL-PHYS-PUB-2018-054, Cepeda:2019klc}.
\begin{itemize}
\item Preparation of inputs to the fit taking into account the increased
luminosity, cross-sections, and improved resolution due to the ATLAS
ITk upgrade.
\item Fitting under different systematic uncertainty assumptions,
producing the table for the publication.
\end{itemize}

\subsection*{Additional academic activities}

\begin{itemize}
\item Co-authorship of a paper on using adversarial training to preserve
physically important variables while optimising the analysis
selection \cite{Windischhofer:2019ltt}.
\item Co-organisation of the 2018 iteration of the MLHEP summer school
in Oxford \cite{MLHEP2018}.
\item Authored a tutorial on using adversarial training of classifiers
at the 2019 MLHEP in DESY \cite{MLHEP2019}.
\item Organisation of a multidisciplinary symposium on the impact of artificial
intelligence on society at Keble College.
\item Co-organisation of the ML+Physics machine learning seminar
series in the Oxford's Department of Physics \cite{ML+physics}.
\end{itemize}
\end{originality}






