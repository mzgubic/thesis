\chapter{Search for the $\hmumu$ decay}

\textit{"The Higgs boson may, or may not, couple to the second generation fermions."}

\vspace{5mm}
\begin{flushright}
--- a bright bunch of ATLAS scientists.
\end{flushright}

\thispagestyle{empty}
\newpage

The interactions between the Higgs boson and the vector bosons and
third generation fermions, critical to the discovery in 2012, soon 
became the preferred way to study the properties of the Higgs boson.
The mass \cite{Aad:2015zhl, Aaboud:2018wps, CMS-PAS-HIG-19-004}
as well as inclusive and differential cross-sections were measured
\cite{Aad_2015, Chatrchyan_2014, Aaboud:2017oem, Sirunyan:2018sgc, Aaboud:2018ezd},
and new production modes, $\tth$ \cite{Sirunyan:2018hoz, Aaboud:2018urx}
and $\vh$ \cite{Sirunyan:2018kst, Aaboud:2018zhk}, have been
observed. However, the interactions between the Higgs boson
and the first or second generation of fermions have not been
observed yet due to experimental challenges. In the leptonic
sector the main challenge is a very small branching fraction due
to the small mass of electron and the muon compared to the tau.
In the quark sector some branching ratios, for example to the charm
quark, are larger than those in which the Higgs was discovered.
However, detection is experimentally difficult in the LHC environment
due to the difficulty in flavour tagging, and estimating and modelling
backgrounds \cite{Aaboud:2018fhh, CMS-PAS-HIG-18-031}. Ultimately the
decay to muon pairs offers the best sensitivity to probe the
interactions between the Higgs boson and the second generation
fermions.






BR in the SM, new physics could modify the branching ratio

past searches by atlas and cms

this result is updated with the new data. mass of the higgs is assumed to be.

brief description of the analysis.

improvements w.r.t. the previous one.





\section{Data and MC simulation samples}

\section{Physics objects}

\section{Event selection}

\section{Event categorisation}

\section{Signal modelling}

\section{Background modelling}

\section{Systematic uncertainties}

\section{Results}


