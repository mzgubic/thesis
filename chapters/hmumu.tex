\chapter{Search for the $\hmumu$ decay}

\textit{"The Higgs boson may, or may not, couple to the second generation fermions."}

\vspace{5mm}
\begin{flushright}
--- a bright bunch of ATLAS scientists.
\end{flushright}

\thispagestyle{empty}
\newpage

The interactions between the Higgs boson and the vector bosons and
third generation fermions, critical to the discovery in 2012, soon 
became the preferred way to study the properties of the Higgs boson.
The mass \cite{Aad:2015zhl, Aaboud:2018wps, CMS-PAS-HIG-19-004},
as well as inclusive and differential cross-sections
\cite{Aad_2015, Chatrchyan_2014, Aaboud:2017oem, Sirunyan:2018sgc, Aaboud:2018ezd},
have been measured and new production modes, $\tth$ \cite{Sirunyan:2018hoz, Aaboud:2018urx}
and $\vh$ \cite{Sirunyan:2018kst, Aaboud:2018zhk}, have been 
observed. However, the interactions between the Higgs boson
and the first or second generation of fermions have not been
observed yet due to experimental challenges. In the leptonic
sector the main challenge is a very small branching fraction due
to the small mass of electron and the muon compared to the tau.
In the quark sector some branching ratios, for example to the charm
quark pair, are larger than those in which the Higgs was discovered.
However, detection is experimentally difficult in the LHC environment
due to the difficulty in flavour tagging, and estimating and modelling
backgrounds \cite{Aaboud:2018fhh, CMS-PAS-HIG-18-031}. Ultimately the
decay to muon pairs offers the best sensitivity to probe the
interactions between the Higgs boson and the second generation
fermions.

In the Standard Model the branching ratio of the Higgs boson with a
mass of 125.09 \GeV~is predicted to be $2.17 \times 10^{-4}$
\cite{deFlorian:2016spz}. This could be modified by physics beyond the
SM \cite{Giudice:2008uua, Dery:2013rta, PhysRevD.80.095023}, meaning
that any deviation from the predicted value could be a sign of new
physics.

ATLAS and CMS experiments have already conducted searches using
the LHC Run 1 data collected at centre-of-mass energies $\sqrts = 7$
and 8 \TeV, both setting the 95\% confidence level upper limits on the
product of the Higgs production cross-section and the branching ratio
to muon pairs \cite{Aad:2014xva, Khachatryan:2014aep} of about seven
times the SM expectation. Using the data collected at $\sqrts = 13$
\TeV, in years 2015 and 2016 the observed upper limit was further
decreased to 2.8 and 2.9 by ATLAS and CMS respectively
\cite{Aaboud:2017ojs, CMS-PAS-HIG-17-019}, with ATLAS
further updating its result to 2.1 using the data collected in 2017
\cite{ATLAS-CONF-2018-026}. This corresponds to $0.9 \sigma$ expected
significance, meaning that the channel is on the verge of evidence.

This chapter presents the search for the Higgs boson decay to muon
pairs using full ATLAS Run 2 dataset collected between 2015 and 2018
at $\sqrts = 13$ \TeV, almost doubling the integrated luminosity of
the dataset since the last result to $139~\ifb$. The Higgs boson mass
is assumed to be 125.09 \GeV~for all presented results.

The overall strategy of the analysis is to select events with
two oppositely charged muons passing the single muon triggers,
and apply additional requirements
to reduce $\ttbar$ and diboson backgrounds. Events are then split
in three channels based on whether they contain zero, one, or two or
more jets in addition to the muon pair. They are further split in
individual categories using a multivariate discriminant, based
on the differences in kinematics between the muon pairs coming
from a Higgs boson produced in the main production modes, and the
muon pairs coming from background events, which are dominated by
the Drell-Yan spectrum. Signal events tend to be more central and
have higher transverse momentum, their jet multiplicity is higher,
and the VBF production mode has a unique signature of two 
back-to-back high-$\pt$ jets with little hadronic activity between
them. Finally, a maximum likelihood fit to the dimuon invariant mass
spectrum is performed in all categories simultaneously, extracting
the signal strength parameter. The analysis is limited by
statistical uncertainty arising from a limited size of the dataset.
The systematic uncertainties are dominated by the uncertainty on
the background modelling, assessed using a dedicated simulated sample.
Additional systematic uncertainties arise from the normalisation 
of signal sample and migration between the categories, such as
the uncertainty on the production cross-section and branching ratio
from the theoretical side, and luminosity, muon momentum resolution
and detector calibration from the experimental side.

A number of improvements were introduced with respect to the
previous result. Selection was improved by including additional
muons from the corners of detector acceptance, by improving
the resolution of the invariant mass by recovering the final
state radiation, and by better rejection of \pileup~ jets in the
forward region. The allocation of events in the categories was
improved by employing a fully multivariate approach in lieu
of a hybrid approach combining a multivariate discriminant and 
cut-based categories. Finally, the background modelling was
improved by using a new functional form and larger DY simulation
dataset. The total effect of all improvements is an approximately
20--30\% increase in the analysis sensitivity and is dominated
by the improvements from using a fully multivariate categorisation
approach.

\section{Data and MC simulation samples}

The proton-proton collision data used in this analysis was
collected in 2015, 2016, 2017, and 2018 at the centre-of-mass
energy of 13 \TeV~in the main physics stream. It corresponds
to an integrated luminosity of 139 $\ifb$ after passing the data
quality checks which ensure that the important parts of the
detector are switched on and function as intended.

The MC simulation samples are used in the analysis for a 
variety of purposes. The signal samples are used to optimise
the event selection and to determine the normalisation and
shape of the signal model in the analysis categories.
Centrally produced background MC simulation samples are used
to optimise the event selection and to validate the muon
momentum resolution in the analysis categories, but not to
determine the normalisation or shape of the backgrounds.
While the normalisation and shape parameters are determined
in the final fit to data, the functional form is selected
in each analysis category independently using a custom
high-statistics MC simulation of Drell-Yan events.

\subsection{Background samples}

The leading irreducible background for the analysis is the
Drell-Yan (DY) production of muon pairs. It is simulated using
\texttt{Sherpa}2.2.1 with the NNPDF3.0 set of parton
distribution functions \cite{Ball:2014uwa} in slices of
$\HT$ and with $c$- and $b$-quark filters, simulating 0--2
jet events at next-to-leading order (NLO) and at leading order
(LO) for 3 and 4 jets. In order to populate the tails of the
DY invariant mass distribution relevant for the analysis an
additional high-statistics sample is generated with identical
setup an an additional cut of $\mmumu > 100$ \GeV. Finally, 
\texttt{Sherpa} is also used to model the electroweak $Z$+jets
process with up to two additional jets at LO accuracy beyond
the first two jets.



\subsection{Signal samples}

\subsection{Custom Drell-Yan samples}


\section{Physics objects}

\section{Event selection}

\section{Event categorisation}

\section{Signal modelling}

\section{Background modelling}

\section{Systematic uncertainties}

\section{Results}


