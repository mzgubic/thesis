\chapter{Search for the $\hmumu$ decay}

\textit{"The Higgs boson may, or may not, couple to the second generation fermions."}

\vspace{5mm}
\begin{flushright}
--- a bright bunch of ATLAS scientists.
\end{flushright}

\thispagestyle{empty}
\newpage

The interactions between the Higgs boson and the vector bosons and
third generation fermions, critical to the discovery in 2012, soon 
became the preferred way to study the properties of the Higgs boson.
The mass \cite{Aad:2015zhl, Aaboud:2018wps, CMS-PAS-HIG-19-004},
as well as inclusive and differential cross-sections
\cite{Aad_2015, Chatrchyan_2014, Aaboud:2017oem, Sirunyan:2018sgc, Aaboud:2018ezd},
have been measured and new production modes, $\tth$ \cite{Sirunyan:2018hoz, Aaboud:2018urx}
and $\vh$ \cite{Sirunyan:2018kst, Aaboud:2018zhk}, have been 
observed. However, the interactions between the Higgs boson
and the first or second generation of fermions have not been
observed yet due to experimental challenges. In the leptonic
sector the main challenge is a very small branching fraction due
to the small mass of electron and the muon compared to the tau.
In the quark sector some branching ratios, for example to the charm
quark pair, are larger than those in which the Higgs was discovered.
However, detection is experimentally difficult in the LHC environment
due to the difficulty in flavour tagging, and estimating and modelling
backgrounds \cite{Aaboud:2018fhh, CMS-PAS-HIG-18-031}. Ultimately the
decay to muon pairs offers the best sensitivity to probe the
interactions between the Higgs boson and the second generation
fermions.

In the Standard Model the branching ratio of the Higgs boson with a
mass of 125.09 \GeV~is predicted to be $2.17 \times 10^{-4}$
\cite{deFlorian:2016spz}. This could be modified by physics beyond the
SM \cite{Giudice:2008uua, Dery:2013rta, PhysRevD.80.095023}, meaning
that any deviation from the predicted value could be a sign of new
physics.

ATLAS and CMS experiments have already conducted searches using
the LHC Run 1 data collected at centre-of-mass energies $\sqrts = 7$
and 8 \TeV, both setting the 95\% confidence level upper limits on the
product of the Higgs production cross-section and the branching ratio
to muon pairs \cite{Aad:2014xva, Khachatryan:2014aep} of about seven
times the SM expectation. Using the data collected at $\sqrts = 13$
\TeV, in years 2015 and 2016 the observed upper limit was further
decreased to 2.8 and 2.9 by ATLAS and CMS respectively
\cite{Aaboud:2017ojs, CMS-PAS-HIG-17-019}, with ATLAS
further updating its result to 2.1 using the data collected in 2017
\cite{ATLAS-CONF-2018-026}. This corresponds to $0.9 \sigma$ expected
significance, meaning that the channel is on the verge of evidence.

This chapter presents the search for the Higgs boson decay to muon
pairs using full ATLAS Run 2 dataset collected between 2015 and 2018
at $\sqrts = 13$ \TeV, almost doubling the integrated luminosity of
the dataset since the last result to $139~\ifb$. The Higgs boson mass
is assumed to be 125.09 \GeV~for all presented results.

The overall strategy of the analysis is to select events with
two oppositely charged muons passing the single muon triggers,
and apply additional requirements
to reduce $\ttbar$ and diboson backgrounds. Events are then split
in three channels based on whether they contain zero, one, or two or
more jets in addition to the muon pair. They are further split in
individual categories using a multivariate discriminant, based
on the differences in kinematics between the muon pairs coming
from a Higgs boson produced in the main production modes, and the
muon pairs coming from background events, which are dominated by
the Drell-Yan spectrum. Signal events tend to be more central and
have higher transverse momentum, their jet multiplicity is higher,
and the VBF production mode has a unique signature of two 
back-to-back high-$\pt$ jets with little hadronic activity between
them. Finally, a maximum likelihood fit to the dimuon invariant mass
spectrum is performed in all categories simultaneously, extracting
the signal strength parameter. The analysis is limited by
statistical uncertainty arising from a limited size of the dataset.
The systematic uncertainties are dominated by the uncertainty on
the background modelling, assessed using a dedicated simulated sample.
Additional systematic uncertainties arise from the normalisation 
of signal sample and migration between the categories, such as
the uncertainty on the production cross-section and branching ratio
from the theoretical side, and luminosity, muon momentum resolution
and detector calibration from the experimental side.

A number of improvements were introduced with respect to the
previous result. Selection was improved by including additional
muons from the corners of detector acceptance, by improving
the resolution of the invariant mass by recovering the final
state radiation, and by better rejection of \pileup~ jets in the
forward region. The allocation of events in the categories was
improved by employing a fully multivariate approach in lieu
of a hybrid approach combining a multivariate discriminant and 
cut-based categories. Finally, the background modelling was
improved by using a new functional form and larger DY simulation
dataset. The total effect of all improvements is an approximately
20--30\% increase in the analysis sensitivity and is dominated
by the improvements from using a fully multivariate categorisation
approach.

\section{Data and MC simulation samples}

The proton-proton collision data used in this analysis was
collected in 2015, 2016, 2017, and 2018 at the centre-of-mass
energy of 13 \TeV~in the main physics stream. It corresponds
to an integrated luminosity of 139 $\ifb$ after passing the data
quality checks which ensure that the important parts of the
detector are switched on and function as intended.

The MC simulation samples are used in the analysis for a 
variety of purposes. The signal samples are used to optimise
the event selection and to determine the normalisation and
shape of the signal model in the analysis categories.
Centrally produced background MC simulation samples are used
to optimise the event selection and to validate the muon
momentum resolution in the analysis categories, but not to
determine the normalisation or shape of the backgrounds.
While the normalisation and shape parameters are determined
in the final fit to data, the functional form is selected
in each analysis category independently using a custom
high-statistics MC simulation of Drell-Yan events.

\subsection{Background samples}

\label{sec:bkg-mc}

The leading irreducible background for the analysis is the
Drell-Yan (DY) production of muon pairs. It is simulated using
\textsc{Sherpa} 2.2.1 with the NNPDF3.0 set of parton
distribution functions (PDF) \cite{Ball:2014uwa} in slices of
$\HT$ and with $c$- and $b$-quark filters, simulating 0--2
jet events at next-to-leading order (NLO) and at leading order
(LO) for 3 and 4 jets. In order to populate the tails of the
DY invariant mass distribution relevant for the analysis an
additional high-statistics sample is generated with identical
setup an an additional cut of $\mmumu > 100$ \GeV. Finally, 
\textsc{Sherpa} is also used to model the electroweak $Z$+jets
process with up to two additional jets at LO accuracy beyond
the first two jets.

With the mass of the top-quark set to $m_t = 172.5$ \GeV, the
$\ttbar$ and single-top samples are generated using
\textsc{Powheg-Box v2} \cite{powheg, Frixione_2007, Alioli_2010}
with NNPDF3.0 PDF set and parton showering and hadronisation
done in \textsc{Pythia} 8.186 using the A14 parameter set
\cite{ATL-PHYS-PUB-2014-021}. $\ttbar$ cross-section is
computed to next-to-next-to-leading order (NNLO) in
QCD with next-to-next-to-leading logarithmic
(NNLL) soft gluon terms taken into account. The cross-section
of the single-top is computed with prescriptions from
\cite{Kidonakis:2011wy, Kidonakis:2010ux}, and all the processes
($t$-channel, $s$-channel, and $Wt$-channel) are generated
separately.

Diboson processes, $WZ$ and $ZZ$, are generated using
\textsc{Sherpa} 2.2.1 with the NNPDF3.0 PDF set, and are normalised
directly to the \textsc{Sherpa} prediction. Only semi-leptonic
and fully-leptonic decays are simulated.

\subsection{Signal samples}

Signal samples are generated for all four main production modes.
The main contribution comes from the $\ggf$ signal sample,
generated with \textsc{Powheg-Box v2} with PDF4LHC15 PDF set and
using next-to-next-to-leading order with parton shower matching
(NNLOPS) \cite{Hamilton:2013fea}, achieving NNLO in QCD after
the re-weighting in rapidity of the Higgs boson. \vbf and
$q\bar{q}/qg \rightarrow \vh$ production modes are generated to NLO
accuracy in QCD using \textsc{Powheg-Box v2} with NNPDF3.0 PDF set.
$gg \rightarrow ZH$ is generated using \textsc{Powheg-Box} at LO
accuracy. $\tth$ production mode is generated using MadGraph5\_aMC@NLO
at NLO accuracy using NNPDF3.0NLO PDF set and the A14 tune.
Parton shower and hadronisation is done using \textsc{Pythia}
8 for all samples, using \textsc{EvtGen} v1.2.0 \cite{LANGE2001152}
for $c$- and $b$-hadron decays. \textsc{Pythia} is also used to
overlay minimum bias events to model the effects of \pileup~for all
simulated events.

The Higgs boson production cross-sections and the branching ratio
to the muon pairs are taken from the CERN Yellow Report 4
\cite{deFlorian:2016spz} at the mass of 125 \GeV. The cross-section
for the \ggf production mode are computed to
next-to-next-to-next-to-leading order (N3LO) \cite{Anastasiou:2016cez}
in QCD with NLO electroweak corrections applied
\cite{Aglietti:2004nj, Actis:2008ug} under the assumptions that
they factorise. VBF cross-section is computed with full NLO QCD
and electroweak corrections \cite{Ciccolini:2007jr, Ciccolini:2007ec,
Arnold:2008rz}, and approximate but highly accurate NNLO QCD
corrections \cite{Bolzoni:2010xr}. \vh~cross-section is computed at
NNLO in QCD \cite{Brein:2003wg} with NLO eletroweak
corrections \cite{Ciccolini:2003jy}.

All signal and background samples are passed through the full ATLAS
detector simulation. Pile-up re-weighting, object momentum
smearing, and efficiency scale factors are applied to the MC
simulation samples in order to make them mimic the collected data.

\subsection{Custom Drell-Yan sample}

Extract the signal strength paramater relies on a
maximum likelihood fit to the invariant mass spectrum in data,
using parametrised models of both signal and background. The choice
of the functional form describing the background is crucial because
a mismodelling of the background results in a bias on the extracted
signal strength. This is particularly significant when the
signal-to-background ratio is very small, because the impact can be
large.

In order to select the functional form and evaluate the resulting
bias, referred to as the ``spurious signal", a fit to the
background-only spectrum is performed. Ideally, this would be done
on the background MC simulation described in Section \ref{sec:bkg-mc},
but because of the high computational cost associated with the 
simulation of the detector response the statistics of the sample
are on the same order of magnitude as the data. As a result, the
the extracted spurious signal value is unreliable and dominated by
the statistical fluctuations. A custom high-statistics sample of DY
events is generated to overcome this challenge.

The custom DY sample needs to have significantly larger statistics
than the data, requiring a much faster generation. To facilitate
this, the sample is produced at a generator-level only, skipping
the simulation of showering, hadronisation, and detector response,
replacing them by parametrised smearing functions. The effect on
the dimuon invariant mass spectrum is minimal, and because of the
parametrised smearing the jet mismodelling is a second order effect.
On the other hand, final state radiation (FSR) does have an effect on
the invariant mass spectrum and is simulated by \textsc{Photos++}
\cite{Golonka:2006tw}. This setup was fast enough to generate 20
billion events corresponding to approximately 100 $\iab$ in the 
regions of parameter space relevant for the analysis. A drawback of
the sample is that it only models the dominant DY part of the
invariant mass spectrum.

At generator level, the samples are produced separately for zero
or one parton and two partons in addition to the muon pair:
\begin{itemize}
\item \textsc{Powheg} is used to generate an inclusive NLO $\zmumu$
sample, modelling zero and one parton in addition to the muon pair.
For efficiency reasons two sub-samples are generated, one with
$60 < \mmumu < 95$ \GeV~($2.5\times10^{9}$ events), and the other
with $\mmumu > 95$ \GeV~($10\times10^{9}$ events) requirement. Even
after detector smearing the latter sample dominates the fit region.
\item \textsc{Alpgen} \cite{Mangano:2002ea} is used to generate
events with two partons in addition to the muon pair at LO. The 
computation uses only the matrix element with two additional
partons and requires both of them to have $\pt > 25$ \GeV~as well
as $\Delta R(j,j) > 0.4$ to minimise the overlap with the
\textsc{Powheg} sample. Similarly to the \textsc{Powheg} sample,
events are generated with $60 < \mmumu < 95$ \GeV~($1.75\times10^{9}$
events) and $\mmumu > 95$ \GeV~($2.5\times10^{9}$ events)
requirements. These events are crucial to populate the analysis
categories with a large fraction of the signal events coming from
the VBF production mode.
\end{itemize}
Both samples are corrected in the same way to model the effects of
the detector. Muon momentum smearing is done using parametrised functions
derived from MC simulation described in Section \ref{sec:bkg-mc}.
Effects from identification, reconstruction, isolation, and impact
parameter efficiencies are also derived from MC simulation and 
applied as weights, while trigger efficiency weights are derived
from data. FSR momentum is smeared using parametrised prescription
from the ATLAS subgroup specialising in electrons and photons.
Photons are dropped randomly to emulate the reconstruction
efficiency. Smearing of jet momenta is performed using the parametrisation
from the ATLAS subgroup specialising in jets and missing energy.
The effect of \pileup~jets is modelled by building a library of
jets from minimum-bias events in MC simulation and overlaying a
number of sampled jets from the library for each event. A dedicated
missing transverse momentum smearing is derived from the Run 2 data.

\section{Physics objects}

\section{Event selection}

\section{Event categorisation}

\section{Signal modelling}

\section{Background modelling}

\section{Systematic uncertainties}

\section{Results}


