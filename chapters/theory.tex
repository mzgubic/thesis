\chapter{The Standard Model}

\textit{"...it turns out that based on their findings, which will be confirmed
or contradicted when the Swiss machine is up and running, turns out there's
slightly more positive than negative muons in all of our atoms, which would
justify the faith of all the believers of the world, make you more
optimistic, and give us an explanation for how we might have all come to this
moment from the primordial slime."}
\vspace{5mm}
\begin{flushright}
--- Bill Clinton, Davos 2011
\end{flushright}

\newpage
Our current best understanding of Nature is that the dynamics of matter is governed
by four fundamental forces. \textit{General Relativity} (GR) provides a
description of spacetime, the arena in which matter exists and interacts, and asserts
that spacetime is not a static entity but is rather dynamically shaped by the presence
of matter. It also explains one of the forces, the gravitational force, as an apparent
phenomenon arising from geodesic motion in curved spacetime. The \textit{Standard Model}
(SM) not only provides a description of the remaining three forces, the electromagnetic force,
the weak force, and the strong force, but also provides a description of matter, and 
elucidates the origin of mass.

\section{Phenomenology}

phenomenological description: particles, forces

The Standard Model is expressed in the language of Quantum Field Theory (QFT). QFT 
extends the domain of quantum mechanics, which characterises the very small, to
the domain of special relativity, which governs the very fast. It does so by
proposing that the fundamental objects are not particles, but rather fields, the
excitations of which are interpreted as particles.

Similarly to classical field theory the system under consideration is characterised
by a Lagrangian density, usually simply called the \textit{Lagrangian}. According to
the principle of least action, the equations of motion for a system are then derived
by solving the Euler-Lagrange equation,
\begin{equation}

\end{equation}

\section{The Higgs mechanism}

mathematical description: QFT, local gauge invariance, the Higgs mechanism, relationship between couplings and masses

\section{Shortcomings}

The success
of the predictive power of the Standard Model is unparalleled in the history of science.
It has been tested to unprecedented levels of numerical precision and even predicted new
fundamental particles before any experimental evidence for their existence. However,
despite its successes, the SM has numerous shortcomings. While it makes exceptionally
precise predictions on a vast number of the results of measurements over many orders
of magnitude, there are a few observations that it is unable to explain at all.
Cosmological observations have firmly established the existence of dark matter via its
gravitational interactions which can not be explained by matter in the SM, and the
existence of dark energy, of which the SM is completely silent. The SM is also unable
to explain the matter-antimatter asymmetry in the universe. From the theoretical
perspective it is unsatisfactory that there is no quantum field description of gravity
in the SM. Additionally, there are a couple of parameters which appear to have taken a
very special value by chance. The bare mass of the Higgs boson is considered to be
unnaturally close to its quantum corrections, and the term containing the strong
charge-parity (CP) violation in the SM Lagrangian is experimentally compatible with
zero. All of these shortcomings suggest that the SM is not a complete description of Nature.


Two complementary experimental strategies to search for physics beyond the SM are
employed at the Large Hadron Collider (LHC), a particle accelerator hosted by the
European Organisation for Nuclear Research (CERN). The first approach directly searches
for yet unobserved phenomena that would reveal new particles. The second approach tests
the predictions of the SM in search for a discrepancy which would signal the direction
in which direct searches should be made. While the results of direct approaches are
more easily interpretable, the advantage of indirect approaches is that they can probe
particles kinematically inaccessible at the LHC. This thesis utilises the second
approach and searches for a decay of the Higgs boson to a muon pair, which is predicted
to be rare by the SM. The LHC accelerates and collides proton beams at the unprecedented
centre-of-mass energy of 13 TeV. The debris of these collisions is recorded by ATLAS, a
general-purpose particle detector, and reconstructed using the compute resources of the
worldwide computing grid. The analysis of the data aims to find a small number of these
rare decays against a large noisy collection of background events.





