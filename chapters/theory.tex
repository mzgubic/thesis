\chapter{Theoretical Introduction}

\textit{"...it turns out that based on their findings, which will be confirmed
or contradicted when the Swiss machine is up and running, turns out there's
slightly more positive than negative muons in all of our atoms, which would
justify the faith of all the believers of the world, make you more
optimistic, and give us an explanation for how we might have all come to this
moment from the primordial slime."}
\vspace{5mm}
\begin{flushright}
--- Bill Clinton, Davos 2011
\end{flushright}

\newpage
\noindent
Our current best understanding of Nature is that the dynamics of matter is governed
by four fundamental forces. \textit{General Relativity} (GR) provides a
description of spacetime, the arena in which matter exists and interacts, and asserts
that spacetime is not a static entity but is rather dynamically shaped by the presence
of matter. It also explains one of the forces, the gravitational force, as an apparent
phenomenon arising from geodesic motion in curved spacetime. The \textit{Standard Model}
(SM) not only describes the remaining three forces, the electromagnetic force,
the weak force, and the strong force, but also provides a description of matter, and 
elucidates the origin of mass.

\section{The Standard Model}

The Standard Model is expressed in the language of Quantum Field Theory (QFT). QFT 
extends the domain of quantum mechanics, which characterises the very small, to
the domain of special relativity, which governs the very fast. It does so by
proposing that the fundamental objects are not particles, but rather fields, the
excitations of which are interpreted as particles.

Similarly to classical field theory the system under consideration is characterised
by a Lagrangian density, usually simply called the \textit{Lagrangian}. According to
the principle of least action, the equations of motion for a system are then derived
by solving the Euler-Lagrange equation \cite{Thomson:2013zua},
\begin{equation}
\partial_\mu \left(\frac{\partial \mathcal{L}}{\partial(\partial_\mu \phi_i)}\right)
- \frac{\partial{\mathcal{L}}}{\partial \phi_i} = 0,
\end{equation}
where $\mathcal{L}$ is the Lagrangian, $\phi_i$ are the fields, partial derivatives
are taken w.r.t. the spacetime coordinates, and Einstein summation convention implies
the sum over repeated indices. The Lagrangian therefore contains all
information about the dynamical system. 

The Lagrangian must be invariant under the transformations which leave the
system it describes physically unchanged. Apart from the Poincar\'e symmetries, which
ensure the physics is invariant w.r.t. translations, rotations, and changes of inertial
reference frame, the defining symmetry of the Standard Model is the internal
$SU(3)_C \times SU(2)_L \times U(1)_Y$ local gauge symmetry.

$SU(3)_C$ group symmetry, where $C$ stands for colour, defines quantum chromodynamics
(QCD), the theory of strong force interactions between quarks and gluons. The electroweak
interactions, a unified theory of the electromagnetic and weak forces, are described by
the $SU(2)_L \times U(1)_Y$ group symmetry. Here, the $L$ refers to left-handed particles
chirality, which is related to the relative direction of particle momentum and spin, and
$Y$ is the weak hypercharge, related to the electric charge and third component of the
weak isospin \cite{Thomson:2013zua}.

The requirement for the Lagrangian to be locally gauge invariant places some constraints
on the allowed terms. Notably, the four-derivative $\partial_\mu$ is replaced by a 
covariant derivative, which introduces the interactions between the matter and gauge
fields. On the other hand, local gauge invariance prohibits the naive mass terms for
both the gauge bosons and fermions \cite{Thomson:2013zua}. This is in stark contrast
with the experimental reality: fermions and weak gauge bosons are massive.

This embarrasing discrepancy is resolved by the Higgs mechanism, which generates the
masses of the fermions and gauge bosons by interactions to the non-zero Higgs field
value at the minimum of the Higgs potential.

\section{Elementary particles}

\section{Electroweak interactions}

A unified theory of quantum electrodynamics (QED) and the weak interaction was developed
by Glashow, Salam, and Weinberg (GSW) in the 1960s \cite{Thomson:2013zua}. 

weak isospin doublet

TODO: introduce $g_Z$

TODO: explain the terms in the covariant derivative

%Local gauge invariance leaves the physics invariant under the arbitrary
%choice of gauge. In the simplest U(1) case this means that the complex phase of the
%singlet fields can be chosen arbitrarily at any point in spacetime. As a result, the
%Standard Model Lagrangian must be invariant under the transformation \cite{Thomson:2013zua}
%\begin{equation}
%\phi(x) \rightarrow \phi'(x) = e^{iq\chi(x)}\phi(x),
%\end{equation}
%of the singlet fields $\phi(x)$ where the phase $\chi(x)$ is \textit{local} because it
%depends on the spacetime position $x$. It is this symmetry which requires the use
%of covariant derivative to make kinetic terms such as 
%
%prohibits the naive mass terms of the form
%\begin{equation}
%lal
%\end{equation}
%
%
%As will be described later in this chapter, it is this symmetry which
%prohibits the naive mass terms for fermions and gauge bosons and calls for an alternative
%mechanism to generate mass.
%
%The detailed description of the Standard Model Lagrangian can be found in standard
%references, such as \cite{Thomson:2013zua, Schwartz:2013pla}, wh

\section{The Higgs mechanism}

In the GSW model, the Higgs model is a weak isospin doublet of complex scalar fields \cite{Thomson:2013zua},
\begin{equation}
\phi = \begin{pmatrix} \phi^+ \\ \phi^0 \end{pmatrix},
\end{equation}
described by the Lagrangian
\begin{equation}
\mathcal{L} = (\partial_\mu \phi)^\dag (\partial^\mu \phi) - \mu^2(\phi^\dag\phi) - \lambda(\phi^\dag \phi)^2,
\label{eq:higgs_lag}
\end{equation}
where the potential $V(x) = \mu^2(\phi^\dag\phi) + \lambda(\phi^\dag \phi)^2$
has an infinite number of minima satisfying $\phi^\dag \phi = -\frac{\mu^2}{2\lambda}$
in the case when $\mu^2 < 0$. The symmetry is spontaneously broken by choosing a
particular minimum from the allowed set, and the fields can be expanded around the chosen minimum.
In the unitary gauge, the Higgs doublet is written as
\begin{equation}
\phi(x) = \frac{1}{\sqrt{2}} \begin{pmatrix} 0 \\ v + h(x) \end{pmatrix},
\end{equation}
where the dependence on $x$ has been made explicit to distinguish the constant
$v$ from the Higgs boson $h(x)$.

In order to make the Lagrangian in Eq. \ref{eq:higgs_lag} invariant under the electroweak
$SU(2)_L \times U(1)_Y$ symmetry, the derivative is replaced by the covariant
derivative \cite{Thomson:2013zua}
\begin{equation}
\partial_\mu \rightarrow D_\mu = \partial_\mu + i g_W \mathbf{T} \cdot \mathbf{W}_\mu
+ i g' \frac{Y}{2} B_\mu.
\end{equation}
The kinetic term in the rewritten Lagrangian, $(D_\mu \phi)^\dag (D^\mu \phi)$, generates the
masses of the gauge bosons. Rearranging, the masses of gauge bosons can be read off as
\begin{equation}
m_W = \frac{1}{2} g_W v,
\end{equation}
\begin{equation}
m_A = 0,
\end{equation}
\begin{equation}
m_Z = \frac{1}{2}v\sqrt{g_W^2 + g'^2}.
\end{equation}
Furthermore, the kinetic term also contains the interaction terms between the gauge bosons
and the Higgs boson $h(x)$. Both the couplings to the $W$, $g_{HWW}=g_W m_W$, and to the $Z$,
$g_{HZZ} = g_Z m_Z$, are proportional to their respective masses \cite{Thomson:2013zua}.

Similarly, naive fermion mass terms of the form $-m \bar{\psi}\psi$ are not allowed because
they violate the $SU(2)_L \times U(1)_Y$ symmetry. However, terms of the form
$\bar{\psi_L}\phi\psi_R$ are allowed, and they give rise to both the fermion mass terms
via the coupling of fermions to the non-zero vacuum expectation value of the Higgs field,
as well as the couplings to the Higgs boson itself. Importantly for this thesis, the theory
predicts a direct relationship between the Yukawa couplings to the Higgs boson and the
masses of fermions,
\begin{equation}
g_f = \sqrt{2}\frac{m_f}{v},
\end{equation}
where $g_f$ is the Yukawa coupling for fermion $f$, $m_f$ is the mass of the fermion $f$,
and $v$ is the vacuum expectation value of the Higgs field \cite{Thomson:2013zua}.

\section{Higgs Phenomenology at the LHC}

Experimental discovery of a new Higgs-like particle was first reported by ATLAS and CMS
experiments in 2012.
In July 2012 the ATLAS and CMS experiments observed a new elementary particle with the
mass of approximately 125 GeV, consistent with the properties of the SM Higgs boson
\cite{Aad:2012tfa, Chatrchyan:2012xdj}. Further studies of spin, parity, and production
and decay rates in both experiments have confirmed that the particle is indeed the
SM Higgs boson \cite{Aad:2015mxa, PhysRevD.92.012004, Aad:2015zhl, Khachatryan:2016vau}.


\section{Shortcomings}

The success
of the predictive power of the Standard Model is unparalleled in the history of science.
It has been tested to unprecedented levels of numerical precision and even predicted new
fundamental particles before any experimental evidence for their existence. However,
despite its successes, the SM has numerous shortcomings. While it makes exceptionally
precise predictions on a vast number of the results of measurements over many orders
of magnitude, there are a few observations that it is unable to explain at all.
Cosmological observations have firmly established the existence of dark matter via its
gravitational interactions which can not be explained by matter in the SM, and the
existence of dark energy, of which the SM is completely silent. The SM is also unable
to explain the matter-antimatter asymmetry in the universe. From the theoretical
perspective it is unsatisfactory that there is no quantum field description of gravity
in the SM. Additionally, there are a couple of parameters which appear to have taken a
very special value by chance. The bare mass of the Higgs boson is considered to be
unnaturally close to its quantum corrections, and the term containing the strong
charge-parity (CP) violation in the SM Lagrangian is experimentally compatible with
zero. All of these shortcomings suggest that the SM is not a complete description of Nature.


Two complementary experimental strategies to search for physics beyond the SM are
employed at the Large Hadron Collider (LHC), a particle accelerator hosted by the
European Organisation for Nuclear Research (CERN). The first approach directly searches
for yet unobserved phenomena that would reveal new particles. The second approach tests
the predictions of the SM in search for a discrepancy which would signal the direction
in which direct searches should be made. While the results of direct approaches are
more easily interpretable, the advantage of indirect approaches is that they can probe
particles kinematically inaccessible at the LHC. This thesis utilises the second
approach and searches for a decay of the Higgs boson to a muon pair, which is predicted
to be rare by the SM. The LHC accelerates and collides proton beams at the unprecedented
centre-of-mass energy of 13 TeV. The debris of these collisions is recorded by ATLAS, a
general-purpose particle detector, and reconstructed using the compute resources of the
worldwide computing grid. The analysis of the data aims to find a small number of these
rare decays against a large noisy collection of background events.





