\chapter*{Summary}
\addcontentsline{toc}{chapter}{Summary}

The Higgs boson is a key element of the Standard Model,
our current best understanding of Nature, and a rich experimental
programme is currently underway to measure its properties
and interactions with other fundamental particles. Any
discrepancy between the experimental data and predictions
from the Standard Model could point the way towards new
physics, potentially addressing the acute theoretical
and experimental problems of the Standard Model.

The experimental programme relies on the ability to
accelerate protons to very high energies using the Large
Hadron Collider at CERN, and collide them at the centre of
general-purpose detectors, ATLAS, and CMS. The work presented
in this thesis relies on the data collected by the ATLAS
detector and focuses on the search for the decay of the
Higgs boson to muon pairs.

Muons are reconstructed in ATLAS by using information
from the inner tracker, muon spectrometer, and the
calorimeter system. Muon momentum is measured to the
precision of a few percent at the scales relevant for
the search and is calibrated in MC simulation by using
the invariant mass peak of the $Z$ boson as a standard
candle. Muon isolation, important for distinguishing
between real prompt muons and fake muons or muons from
secondary vertices, is described using information from
both the tracker and the calorimeters. The isolation
efficiency in MC simulation is corrected by the scale
factors, measured using the tag-and-probe method. A
novel method of background subtraction reduces the
systematic uncertainties associated with the
scale factors by up to 75\% in the low transverse
momentum regime. A machine learning approach to
estimating and eventually improving muon resolution
with the goal of improving analysis sensitivity is
presented but the observed improvements are too small
to justify the additional complexity of the analysis.

The search for the $\hmumu$ decay proceeds by selecting
events which pass the loosest available unprescaled
single muon triggers and have two oppositely charged
muons. To reduce backgrounds from $\ttbar$ and diboson
backgrounds events with a $b$-tagged jet are rejected.
The selected events are then split based on the number
of jets in the event and further categorised using a 
gradient boosting classifier. The classifiers are trained
with MC simulation and data sidebands using a four-fold
validation approach to avoid overtraining. The boundaries
defining the categories are determined using an exhaustive
grid search maximising the significance proxy combined
across the categories. Signal models are fit separately
in all categories to determine the shape of the signal.
The background model is selected using a dedicated
high-statistics MC simulation of Drell-Yan events, which
is also used to determine the spurious signal systematic
uncertainty. Finally, a combined maximum likelihood fit
is performed simultaneously in all categories to
extract the signal strength and the associated confidence
interval, as well as the discovery significance of the
decay. The data is found to be compatible with both the
Standard Model signal and no signal at all, with the
result being limited by the statistical uncertainty on
the data.

Finally, sensitivity projection studies are presented
that estimate the analysis sensitivity achievable
with future datasets. The analysis is on the verge of
sensitivity to obtain evidence for the decay of the 
Higgs boson to muon pairs, and exciting results are
expected from both ATLAS and CMS experiments in the 
immediate future.






